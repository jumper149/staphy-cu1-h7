\documentclass[a4paper,12pt]{scrartcl}

\usepackage[utf8]{inputenc}
\usepackage[english]{babel}
\usepackage[T1]{fontenc}
\usepackage{fancyhdr}
\usepackage{amsmath}
\usepackage{mathtools}
\usepackage{latexsym}
\usepackage{tensor}
\usepackage{graphicx}
\usepackage{tikz}
\usepackage{hyperref}
\usepackage{booktabs}
\usepackage{textcomp}

\allowdisplaybreaks

\title{Statistical Physics}
\subtitle{Homework, Sheet 4}
\author{Felix Springer --- 10002537}
\date{\today}

\begin{document}

\maketitle

\section{Thermodynamics of a chain [H7]}
In my solution I am using the approach that is shown in the hints.
I am calculating the probability density $P$ in form of a histogram and then try to derive the force $F$ directly from there.

\subsection{Theoretical background}
First I use the relation (\ref{eq:F1}) of force $F$ and pressure $p$, in which I then insert the free Energy $E$, which leads to the partition function $Z$.
\begin{align}
	F &= \int p \text{ d}A \label{eq:F1} \\
	&= \int \underbrace{- \left( \frac{\partial E}{\partial V} \right)_{\tau}}_{= p} \text{d}A \notag \\
	&= \int - \partial_V \underbrace{(- \tau \ln(Z))}_{= E} \text{ d}A \notag \\
	&= \tau \int \partial_V \ln(Z) \text{ d}A \notag \\
	F &= \tau \text{ } \partial_L \ln(Z) \label{eq:F2}
\end{align}

Now with equation (\ref{eq:F2}) I only need to be able to calculate the partition function $Z$.
In this case there is a relation (\ref{eq:P1}) to the probability density $P$.
\begin{align}
	P (N_\nu, \epsilon_\nu) &= \frac{1}{Z} \underbrace{\exp \left( {\frac{\mu N_\nu - \epsilon_\nu}{\tau}} \right)}_{\eqqcolon \alpha} \label{eq:P1} \\
	\implies \text{ } Z &= \frac{\alpha}{P} \label{eq:Z1}
\end{align}

Now I can use equation (\ref{eq:Z1}) to further simplify equation (\ref{eq:F2}).
\begin{align}
	F &= \tau \partial_L \ln \left( \frac{\alpha}{P} \right) \notag \\
	&= \tau ( \underbrace{\partial_L \ln(\alpha)}_{= 0} - \partial_L \ln(P) ) \notag \\
	F &= \tau \text{ } \partial_L \ln(P) \label{eq:F3}
\end{align}

With equation (\ref{eq:F3}) I can directly compute the force $F$ from the probability density $P(L)$ which I am simulating with a distribution of ''randomwalks''.

\end{document}
